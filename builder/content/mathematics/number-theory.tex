\subsection{Calculation of $a^b \mod m$}
if $b \ge \phi(m)$, then value $a^b \equiv a^{\left[b \mod \phi(m)\right] + \phi(m)} \pmod m$.

\subsection{Generators}
A generator exists only for $n = 1, 2, 4, p^k, 2p^k$ for odd primes $p$ and positive integers $k$. 

$g$ is a generator modulo $n$ if any number coprime with $n$ can be represented as $\left[ g^i \mod n \right], 0 \le i < \phi(n)$.

To find a generator:
\begin{itemize}
\item find $\phi(n)$ and $p_1, ..., p_m$ --- the prime factors of $\phi(n)$
\item $g$ is generator only if $g^{\frac{\phi(n)}{p_j}} \not\equiv 1 \pmod n$ for each $j$
\item check $g = 2, 3, 4, ..., p - 1$
\end{itemize}

\subsection{Wilson's theorem}
$p$ is prime if and only if $(p - 1)! \equiv (p - 1) \pmod p$.

\subsection{Quadratic residues}
$q$ is a quadratic residue modulo $p$ if there exists an integer $x$ such that $x^2 \equiv q \pmod p$.
If $p$ is odd prime then there exist $\frac{p + 1}{2}$ residues (including 0).

\subsection{Number theory functions}
\fbox{\begin{minipage}{0.31\textwidth}
\begin{align*}
& \text{For } \; n = p_1^{\alpha_1} \cdot \dots \cdot p_k^{\alpha_k}\\
& \phi(n) = \prod p_i^{\alpha_i - 1} (p_i - 1) - \text{ number of coprime } \le n\\
& F(n) = \frac{n \cdot \phi(n)}{2} - \text{ sum of coprime } \le n, \; for \; n > 1\\
& \mu(n) = (-1)^k \text{ if } \max(\alpha_i) = 1 \text {, else } 0\\
& \sigma_k(n) = \sum_{d|n} d^k\\
& \sigma_0(n) = \prod (\alpha_i + 1)\\
& \sigma_{k > 0}(n) = \prod \frac{p_i^{(\alpha_i + 1) \cdot k} - 1}{p_i^k - 1} 
\end{align*}
\end{minipage}}

\vspace{35pt}

\subsection{Möbius}
\fbox{\begin{minipage}{0.31\textwidth}
\begin{align*}
& g(n) = \sum_{d|n}f(d) \iff f(n) = \sum_{d|n}\mu(d)g\left(\frac{n}{d}\right)\\
& \sum_{n = 1}{x} M\left(\left\lfloor\frac{x}{n}\right\rfloor\right) = 1 \; \text{, where} \; M(n) = \sum_{k=1}^{n}\mu(k)\\
& \sum_{d|n} \phi(d) = n \hspace{12mm} \sum_{d|n} \mu(d) = [n = 1]
\end{align*}
\end{minipage}}