\subsection*{FFT tricks}
\subsubsection*{Two-dimensional FFT}
The complexity is $O(nm(\log n + \log m))$.
The main problem is to resize the matrix. You must add non-empty vectors.

\subsubsection*{Divide-and-conquer FFT}
Suppose we have the following DP relation: $f(t) = g(t) - \sum_{0 \le u < t} f(u) h(t-u)$,
where $g(t)$ and $h(t)$ are known and we want to compute $f(t)$. We can apply divide-and-conquer FFT.

Let $m = \lfloor\frac{l+r}{2}\rfloor$. We guarantee the following invariant conditions.

By the time we compute the values for the segment $[l,r)$, the following conditions are already met:
\begin{itemize}
\item The values for $[0,l)$ on the DP is already determined.
\item The sum of contributions from $[0,l)$ through $[l,r)$ is already applied to the DP in $[l,r)$.
\end{itemize}

When calculate the values for the segment $[l, r)$ do:
\begin{itemize}
\item Calculate the values for the segment $[l,m)$ recursively.
\item Calculate the contributions from $[l,m)$ to $[m,r)$.
\item Calculate the values for the segment $[m,r)$ recursively.
\end{itemize}

\subsection*{Properties of the discrete Fourier transform}

$$DFT(x)_k = \sum_{n = 0}^{N - 1} x_n \cdot e^{-i 2 \pi \frac{k}{N}n}$$

Let $x^R_n = x_{N - n \mod N}$.

$DFT(x^R) = \overline{DFT(x)}.$

For real $x$, $DFT(x)^R = \overline{DFT(x)}$.