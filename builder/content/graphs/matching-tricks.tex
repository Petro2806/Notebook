\subsection*{Minimum cut}
To find the min-cut, search from vertex $S$ on unsaturated edges. 
Original edges from used vertices to unused ones are in the min-cut.

\subsection*{Minimum vertex cover}
The vertex cover problem is not NP-complete in bipartite graphs. 
The minimum number of vertices required to cover all \textbf{edges} is equal to the size of the maximum matching. 
To reconstruct the minimum vertex cover, create a directed graph:

\begin{itemize}
\setlength\itemsep{0em}
\item matched edges from the right part to the left part
\item unmatched edges from the left part to the right part.
\end{itemize}

Start traversal from unmatched vertices in the left part.
The cover includes vertices from the matching:
\begin{itemize}
\setlength\itemsep{0em}
\item unvisited vertices in the left part
\item visited vertices in the right part.
\end{itemize}

\subsection*{Maximum independent set}
The independent set problem is not NP-complete in bipartite graphs. 
It is the complement of the minimum vertex cover.

\subsection*{Minimum edge cover}
A minimum edge cover can be found in \textbf{any} graph. 
The minimum number of edges required to cover all vertices can only be determined in graphs without isolated vertices.
By utilizing one edge in the matching, we cover two vertices, while any other vertices are covered using one edge for each.

\subsection*{DAG paths}
In a DAG, you can find the minimum number of non-intersecting paths that cover all vertices. 
Duplicate vertices and create a bipartite graph with edges $u_L \rightarrow v_R$. 
Edges in the matching correspond to edges in the paths.

\subsection*{Dominating set}
A dominating set for a graph is a subset $D$ of $V$ such that any vertex is in $D$, or has a neighbor in $D$.
The dominating set problem is NP-complete \textbf{even on bipartite graphs}.
It can be found greedily on a tree.