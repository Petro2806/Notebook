\section{Mathcing tricks}
\textbf{Min cut}
To find the min-cut use search from vertex S on not saturated edges. 
Original edges from used vertices to unused is in min-cut.

\textbf{Min vertex cover}
A min vertex cover is not NP-complete in bipartite graphs. 
The minimum number of the vertex to cover all \textbf{edges} is equal to the size of matching. 
To restore min vertex cover, make a directed graph.

\begin{itemize}
\setlength\itemsep{0em}
\item matched edges direct from R to L
\item unmatched edges direct from L to R
\end{itemize}

From unmathced vertices in left part start traversal.
Cover have vertices from matching:
\begin{itemize}
\setlength\itemsep{0em}
\item unvisited vertices in L 
\item visited vertices in R
\end{itemize}

\textbf{Max independent set}
A max independent set is not NP-complete in bipartite graphs. 
It is the complement of the min vertex cover.

\textbf{Min edge cover}
A min edge cover can be found in \textbf{ANY} graphs. 
Minimum edges to cover all vertices are possible to find only in graphs without isolated vertices. 
Using one edges in the matching we cover two vertices, 
and any other vertices we cover using one edge for each. 

\textbf{DAG pathes}
In DAG you can find a minimum number of non-intersecting paths that cover all vertices. 
Duplicate vertices and make a bipartite graph with edges $u_L \rightarrow v_R$. 
Edges in the matching are edges in paths.

\textbf{Dominating set}
Dominating set for a graph G = (V, E) is a subset D of V such that every
vertex not in D is adjacent to at least one member of D. Finding a dominating set
is NP-complete \textbf{even on bipartite graphs}.
Can be found greedily on a tree.

\textbf{Tutte's matrix}

For any graph:
\[
T_{ij} = \begin{cases} 
      \text{rand()} \cdot \text{sgn}(i - j) & (i, j) \in E \\
      0 & \text{otherwise}
   \end{cases}
\]
$det(T) = 0 \iff \text{there is no perfect matching}$


\bigskip 
