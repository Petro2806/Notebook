\section*{Dilworth’s theorem}

A partially ordered set is a set $S$ with a relation $\le$ on $S$ satisfying:
\begin{enumerate}
\item $a \le a$ for all $a \in S$ (reflexivity);
\item if $a \le b$ and $b \le a$, then $a = b$ (antisymmetry);
\item if $a \le b$ and $b \le c$, then $a \le c$ (transitivity).
\end{enumerate}

Dilworth's theorem states that, in any finite partially ordered set, 
the \textbf{largest antichain} has the same size as the \textbf{smallest chain} decomposition. 
Here, the size of the antichain is its number of elements, and the size of the chain decomposition is its number of chains.
