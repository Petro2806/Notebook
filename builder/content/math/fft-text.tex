\textbf{\huge{FFT tricks}}\\
\textbf{FFT in 2D}
The complexity is $O(nm(\log n + \log m))$.
The main problem to resize the matrix. You must add non empty vectors.

\textbf{Divide-and-Conquer FFT}\\
Using D-and-C to calculate DP table. (For example $DP[i] = sum(DP[j] \cdot DP[i - j])$) 

By the time we compute the values for the segment $[l,r)$, the following conditions are already met:
\begin{itemize}
\item The values for $[0,l)$ on the DP table is already determined.
\item The sum of contributions from $[0,l)$ through $[l,r)$ is already applied to the DP table in $[l,r)$.
\end{itemize}

When calculate the values for the segment $[l, r)$ do:
\begin{itemize}
\item Calculate the values for the segment $[l,m)$ recursively.
\item Calculate the contributions from $[l,m)$ to $[m,r)$.
\item Calculate the values for the segment $[m,r)$ recursively.
\end{itemize}

\textbf{DFT facts}\\
DTFT of a convolution $c_k = \sum_{(i + j)\%n=k}a_ib_j$ is DFT.
\fbox{
\begin{minipage}{0.31\textwidth}
\begin{align*}
& DFT(x)_k = \sum_{n = 0}^{N - 1} x_n \cdot e^{-i2\pi\frac{kn}{N}} \hspace{14mm} DFT(x^R) = \overline{DFT(x)}\\
& DFT(x_{n - m})_k = DFT(x)_k \cdot e^{\frac{-i2\pi km}{N}} \hspace{3mm} DFT(x^R) = DFT(x)^R\\
& DFT^{-1}(x) = \frac{1}{N}DFT(x^R) \hspace{17mm} DFT(\overline{x}) = \overline{DFT(x)}^R\\
& DFT(Re(x)) = \frac{1}{2}(DFT(x) + \overline{DFT(x)}^R)\\
& DFT(Im(x)) = \frac{1}{2i}(DFT(x) - \overline{DFT(x)}^R)\\
& DFT(\frac{1}{2}(x + \overline{x}^R)) = Re(DFT(x))\\
& DFT(\frac{1}{2i}(x - \overline{x}^R)) = Im(DFT(x))
\end{align*}
\end{minipage}
}
