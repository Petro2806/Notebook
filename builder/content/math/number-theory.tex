\textbf{Calculation of $a^b \mod m$}\\
if $b \ge \phi(m)$ then value $a^b \mod m$ equals to the value $a^{\left[b \mod \phi(m)\right] + \phi(m)} \mod m$.

\textbf{Generators}\\
Generator exist only for $n = 1, 2, 4, p^k, 2p^k$ for odd primes $p$ and positive integer $k$. 

$g$ is generator for modulo $n$ if any comprime with $n$ can be represented as $\left[ g^i \mod n \right], 0 \le i < \phi(n)$.

To find generator:
\begin{itemize}
\item find $\phi(n)$ and $p_1, ..., p_m$ --- prime factors of $\phi(n)$
\item g is generator only if $g^{\frac{\phi(n)}{p_j}} \mod n \ne 1$ for each $j$
\item check $g = 2, 3, 4, ..., p - 1$
\end{itemize}


\textbf{Wilson}\\
$p$ is prime if and only if $(p - 1)! = (p - 1) \mod p$.

\textbf{Quadratic residue}\\
$q$ is quadratic residue modulo $p$ if there exist integer $x$ that $x^2 = q \mod p$.
If $p$ is odd prime then there exists $\frac{p + 1}{2}$ residues (including 0).

\textbf{Legendre symbol} is equal to 0 if $q$ is divisible by $p$, 
equal to 1 if $q$ is quadratic residue, and -1 otherwise:\\
$\Big(\frac{q}{p}\Big) = q^{\frac{p - 1}{2}} (mod p)$

\textbf{Jacobi symbol} (Legendre symbol for all $p$):\\
$\Big(\frac{q}{p}\Big) = \prod{\Big(\frac{q}{p_i}\Big)^{\alpha_i}}$

\textbf{Number theory functions}
\fbox{\begin{minipage}{0.31\textwidth}
\begin{align*}
& For \; n = p_1^{\alpha_1} \cdot \dots \cdot p_k^{\alpha_k}\\
& \phi(n) = \prod p_i^{\alpha_i - 1} (p_i - 1) - \text{ number of coprime } \le n\\
& F(n) = \frac{n \cdot \phi(n)}{2} - \text{ sum of coprime } \le n, \; for \; n > 1\\
& \mu(n) = (-1)^k \text{ if } \max(\alpha_i) = 1 \text {, else } 0\\
& \sigma_k(n) = \sum_{d|n} d^k\\
& \sigma_0(n) = \prod (\alpha_i + 1)\\
& \sigma_{k > 0}(n) = \prod \frac{p_i^{(\alpha_i + 1) \cdot k} - 1}{p_i^k - 1} 
\end{align*}
\end{minipage}}


\textbf{Mobius}
\fbox{\begin{minipage}{0.31\textwidth}
\begin{align*}
& g(n) = \sum_{d|n}f(d) \iff f(n) = \sum_{d|n}\mu(d)g(\frac{n}{d})\\
& \sum_{n = 1}{x} M(\lfloor\frac{x}{n}\rfloor) = 1 \; where \; M(n) = \sum_{k=1}^{n}\mu(k)\\
& \sum_{d|n} \phi(d) = n \hspace{12mm} \sum_{d|n} \mu(d) = [n == 1]
\end{align*}
\end{minipage}}


\textbf{Burnside's lemma}

Let $G$ be a finite group that acts on a set $X$.

The \textit{orbit} of an element $x$ in $X$ is the set of elements
in $X$ to which $x$ can be moved by the elements of $G$.
The orbit of $x$ is denoted by $G \cdot x$:

 \[G \cdot x = \{g \cdot x\, |\, g \in G\}.\]

For each $g$ in $G$, let $X^g$ denote the set of elements
in $X$ that are fixed by $g$ (also said to be left invariant by $g$),
that is, $X^g = \{ x \in X\, |\, g \cdot x = x \}$.
Burnside's lemma asserts the following formula for the number of orbits,
denoted $|X/G|$:

\[|X/G| = \frac{1}{|G|} \sum_{g \in G} |X^g|.\]
