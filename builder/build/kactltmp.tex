\kactlref{gaussian-integer.hpp}
\defdescription{$n = am + b, \frac{n}{m} = a, n \% m = b$.
use \_\_gcd instead of gcd.\\
\textbf{Facts}: Primes of the form $4n + 3$ are Gaussian primes.
Uniqueness of prime factorization.}
\rightcaption{cb938e, 41 lines}
\begin{lstlisting}[caption={gaussian-integer.hpp}, language=C++]
LL closest(LL u, LL d)
{
	if(d < 0)
		return closest(-u, -d);
	if(u < 0)
		return -closest(-u, d);
	return (2 * u + d) / (2 * d);
}
struct num : complex<LL>
{
	num(LL a, LL b = 0) : complex(a, b) {}
	num(complex a) : complex(a) {}
	num operator/ (num x)
	{
		num prod = *this * conj(x);
		LL D = (x * conj(x)).real();

		LL m = closest(prod.real(), D);
		LL n = closest(prod.imag(), D);

		return num(m, n);
	}
	num operator% (num x)
	{
		return *this - x * (*this / x);
	}
	bool operator == (num b)
	{
		FOR(it, 0, 4)
		{
			if(real() == b.real() && imag() == b.imag())
				return true;
			b = b * num(0, 1);
		}
		return false;
	}
	bool operator != (num b)
	{
		return !(*this == b);
	}
};
\end{lstlisting}
